\Chapter{Hasonló alkalmazások és koncepció}

\Section{Mesterséges intelligencia a videojátékokban}
A mesterséges intelligencia már több mint fél évszázada foglalkoztatja az embereket, mind a tudományos fantasztikus regényekben, mind az informatikában, ambár az utóbbi az viszonylag újkeletű dolog. Az első definíció, és maga a fogalom mint mesterséges intelligencia (artificial intelligence) egy amerikai informatikustól John McCarthytól származik, 1956ból.
%ref: https://courses.cs.washington.edu/courses/csep590/06au/projects/history-ai.pdf

Videojátékok terén is az 50es évekre tehető a kezdet, 1950ben mutatták be a "Bertie the Brain" nevezetű számítógépet, mely segítségével mesterséges intelligencia ellen lehetett játszani a Tic-tac-toe nevezetű játékot. Egy évre rá 1951ben kiadták a Brit Nimrod nevezetű számítógépet ahol szintén gépi ellenfél ellen lehetett megmérkőzni, de ezúttal a Nim nevű játékban.
Ezek a számítógépek elég kezdetlegesek voltak nem is biztos hogy mára már hagyományos értelemben véve videojátéknak neveznénk őket, de kétségkívül az elsők között voltak, akik MI-t használtak ilyen célra.
%ref: https://en.wikipedia.org/wiki/Bertie_the_Brain
%ref: https://en.wikipedia.org/wiki/Nimrod_(computer)

Az első kísérletek után nem kellett sokat várni hogy az eddiginél komolyabb szoftvereket tudjanak írni, erősebb gépi ellenfelekkel, amelyek az évtized végére, és a 60as évek elejére még tovább fejlődtek. Ugyanakkor a 70es évekig kellett várni olyan nagy címekre, mint a \textit{Spacewar!} és a \textit{Pong}, majd az évtized végén megjelenő \textit{ Space Invaders (1978)} és \textit{Pacman (1980)}, melyeknél már fokozatosan erősödtek a gépi ellenfelek.
A 80as és 90es években már szinte minden zsánerben fellelhető volt valamiféle mesterséges intelligencia, legyen szó verekedős, stratégiai, sport, vagy szerepjátékról.

Természetesen sokakban kritika is megfogalmazódott, hogy nem-e úgy vannak leprogramozva a gépi ellenfelek hogy csaljanak, és gyakran ez igaz is volt, mert egyszerűbb egy stratégiai játékban olykor olykor megadni az MI-nek a valódi játékos pozicíóját, mint leprogramozni ténylegesen hogy megtalálja. Gyakori szokás még szintén stratégiai játékokban hogy valamennyi százalék bónuszt kap egy-egy nyersanyagra az ellenfél. Habár a kritika teljesen jogos, és néhol tényleg látványos ez a csalás, manapság sokkal kifinomultabbab ezek a rendszerek, és a lehető legjobb játélélményt szolgálják, mivel szinte elengedhetetlenné vált a videojáték iparban a mesterséges intelligencia használata.

Felmerül ugyanakkor a kérdés, hogy amennyiben egy videojátékba be lehet programozni egy mesterséges intelligenciát ami ellenfélnek (vagy csapattársnak) szolgál az emberi játékos számára, meg lehet-e tanítani vagy be lehet-e programozni egy másik MI-t arra hogy játszon egy játékkal? A továbbiakban ezzel a kérdskörrel fogunk foglalkozni.
\iffalse
\Section{Ember a gép ellen}
aaaaa

\Section{AlphaStar}
Talán a 
\fi