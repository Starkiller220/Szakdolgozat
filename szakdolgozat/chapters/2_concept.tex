\Chapter{Hasonló alkalmazások és koncepció}

\Section{Mesterséges intelligencia a videojátékokban}
A mesterséges intelligencia már több mint fél évszázada foglalkoztatja az embereket, mind a tudományos fantasztikus regényekben, mind az informatikában, ambár az utóbbi az viszonylag újkeletű dolog. Az első definíció, és maga a fogalom mint mesterséges intelligencia (artificial intelligence) egy amerikai informatikustól John McCarthytól származik, 1956ból.
%ref: https://courses.cs.washington.edu/courses/csep590/06au/projects/history-ai.pdf

Videojátékok terén is az 50es évekre tehető a kezdet, 1950ben mutatták be a "Bertie the Brain" nevezetű számítógépet, mely segítségével mesterséges intelligencia ellen lehetett játszani a Tic-tac-toe nevezetű játékot. Egy évre rá 1951ben kiadták a Brit Nimrod nevezetű számítógépet ahol szintén gépi ellenfél ellen lehetett megmérkőzni, de ezúttal a Nim nevű játékban.
Ezek a számítógépek elég kezdetlegesek voltak nem is biztos hogy mára már hagyományos értelemben véve videojátéknak neveznénk őket, de kétségkívül az elsők között voltak, akik MI-t használtak ilyen célra.
%ref: https://en.wikipedia.org/wiki/Bertie_the_Brain
%ref: https://en.wikipedia.org/wiki/Nimrod_(computer)

Az első kísérletek után nem kellett sokat várni hogy az eddiginél komolyabb szoftvereket tudjanak írni, erősebb gépi ellenfelekkel, amelyek az évtized végére, és a 60as évek elejére még tovább fejlődtek. Ugyanakkor a 70es évekig kellett várni olyan nagy címekre, mint a \textit{Spacewar!} és a \textit{Pong}, majd az évtized végén megjelenő \textit{ Space Invaders (1978)} és \textit{Pacman (1980)}, melyeknél már fokozatosan erősödtek a gépi ellenfelek.
A 80as és 90es években már szinte minden zsánerben fellelhető volt valamiféle mesterséges intelligencia, legyen szó verekedős, stratégiai, sport, vagy szerepjátékról.

Természetesen sokakban kritika is megfogalmazódott, hogy nem-e úgy vannak leprogramozva a gépi ellenfelek hogy csaljanak, és gyakran ez igaz is volt, mert egyszerűbb egy stratégiai játékban olykor olykor megadni az MI-nek a valódi játékos pozicíóját, mint leprogramozni ténylegesen hogy megtalálja. Gyakori szokás még szintén stratégiai játékokban hogy valamennyi százalék bónuszt kap egy-egy nyersanyagra az ellenfél. Habár a kritika teljesen jogos, és néhol tényleg látványos ez a csalás, manapság sokkal kifinomultabbab ezek a rendszerek, és a lehető legjobb játélélményt szolgálják, mivel szinte elengedhetetlenné vált a videojáték iparban a mesterséges intelligencia használata.

Felmerül ugyanakkor a kérdés, hogy amennyiben egy videojátékba be lehet programozni egy mesterséges intelligenciát ami ellenfélnek (vagy csapattársnak) szolgál az emberi játékos számára, meg lehet-e tanítani vagy be lehet-e programozni egy másik MI-t arra hogy játszon egy játékkal? A továbbiakban ezzel a kérdskörrel fogunk foglalkozni.

\Section{Ember a gép ellen}
%aaaaa Alphago!

\subsection{AlphaStar}
Talán a legjobb eredményt a Deepmind mesterséges intelligenciája érte el a StarCraft II nevű játékban, mely jelenleg jobb a játékosok 99,8\%-nál. 
%ref:https://www.theverge.com/2019/10/30/20939147/deepmind-google-alphastar-starcraft-2-research-grandmaster-level
Ez a játék talán a legkomplexebb a jelenlegi stratégiai játékok közül, nem hiába fektetnek komoly pénzeket a verseny szintű mérkőzések lebonyolítására.
Léteztek már az AlphaStar előtt is MI-k amik képesek voltak valamilyen szinten játszani, de azok valamiféle előnyt kaptak a cél elérése érdekében, például egyszerűsített pályák, korlátozott játékmenet, emberfeletti képességek, így azok teljesítménye közel sem éri el az elvárt szintet.

Nagy probléma továbbá leküzdeni azt, hogy egyszerűbb játékokkal szemben (mint például a Tic-tac-toe vagy a kő-papír-olló) nincsen úgymond "legjobb stratégia", így rengeteg esetet kell vizsgálni és folyamatosan új módszereket keresni. Ugyanígy a már említett táblás játékokhoz való hozzáállást sem lehet alkalmazni, mert amíg a sakkban és a goban minden információ adott, egy ilyen stratégiai játékban rengeteg a rejtett, ismeretlen tényező, amelyet fel kell fedezni. Mindezt nehezíti hogy ezt valós időben, vagy legalábbis minimális késlekedéssel kell elemezni és döntést hozni, ráadásul figyelembe venni a hosszú távú tervezést is. Ugyanígy a sakkal ellentétben, ahol csak 6 féle bábú van, és egyszerre maximum 32 játszik egy adott mérkőzésen, egy videojátékban több tucat egység variáns lehet, amelyekből több száz is létezhet egyidejűleg, ezzel növelve a komplexitást.

\Section{A StarCraft és a Warcraft}
Mivel mind a két játékot a Blizzard készítette, rengeteg hasonlóság fedezhető fel bennük, ha mind a kettőből az eredeti, első részt vesszük, rájövünk hogy igazából a Starcraft az csak egy továbfejlesztett Warcraft másolat, sci-fi köntösbe bújtatva, és a fajok közötti különbséget már nem csak kinézetben kereshetjük, hanem az egységek statisztikái, és a különböző fajok játékstílusa is más. Ezt a legjobban egy táblázattal lehetne szemléltetni.
\begin{table}[h]
    \centering
    \caption{Összehasonlítás}
    \label{tab:osszehasonlitas}
    \begin{tabular}{l|c|c|}
    ~ & Warcraft & Starcraft \\
    \hline
    Megjelenés & 1994 & 1998 \\
    Játszható fajok száma & 2 & 3 \\
    Fajonként eltrérő egység statisztika & nem & igen \\
    Játszható pályák száma & 20 & 86 \\
    \hline
    \end{tabular}
\end{table}

Látható hogy már a régebbi StarCraft is hatványozottabban komplexebb volt az spirituális elődjénél, így érthető hogy az utódhoz évekbe telt olyan intelligenciát írni, ami szinte mindekit legyőz.

\Section{A Terv}

Mindazok után hogy megnéztük az MI kialakulását a videojátékok terén, továbbá a hasonló alkalmazások jellemzőit, eredményeit. Ideje tervet készíteni, milyen lépések szükségesek ahhoz hogy implementálni lehessen egy Warcraft-al játszó mesterséges intelligenciát.

\subsection{1. lépés: Kommunikáció a DosBox-al}

