\Chapter{Megvalósítás}

Itt történik az elkészült szoftver megvalósításának bemutatása, az előző fejezetben leírt terv szempontjai alapján.

\Section{DosBox}

\subsection{Buildelés}
Első lépésként a forráskódot kell beszerezni, ez a beszerzés időpontjában elérhető legfrissebb verzióban történt meg, a további frissítése új verziókra csak abban az esetben indokolt ha kritikus hibát találok a fejlesztés során, egyébként csak komplikációk adódnának miatta.
% ref: https://sourceforge.net/p/dosbox/code-0/4392/tree/dosbox/

A következő lépés a beszerzett forráskód buildelése. Mivel jelenleg a Windowst használom gyakrabban, így a DosBox buildelése, és a mesterséges intelligencia is Windows alatt fog készülni. Nem lehetetlen hogy Linux alatt is elindulni mind a Dosbox, mind az MI, de a szakdolgazat nem foglalkozik erre az operációs rendszerre történő optimalizálásra.
% ref: https://www.dosbox.com/wiki/Building_DOSBox_with_Visual_Studio

Ha megpróbálnánk így lefordítani, azonnal hibába ütköznénk, ugyanis szükségesek hozzá különböző könyvtárak.

\begin{itemize}

    \item SDL (Simple DirectMedia Layer)
    
    Az egyetlen kötelező könyvtár és talán legfontosabb. Az SDL egy több platformos fejlesztői könyvtár, amelyet arra találtak ki, hogy alacsony szintő hozzáférést biztosítson többek között az egérhez, billentyűzethez, különböző játékvezérlőkhöz, magához a hanghoz OpenGL és Direct3D segítségével. Rengeteg híres videojáték használja ezt a technológiát. Sajnos a DosBox elég válogatós, így a leírás alapján nem a legfirsebb verziót kéri, hanem egy jóval régebbi kiadást.
    \item zlib / libpng (opcionális)
    
    Egy opcionális könyvtár képek és videók mentésére. A tervben vázoltak alapján arra következtetünk, hogy ez még a hasznunkra vállhat, így mind a két könyvtárat is le kell töltenünk, mivel a \textit{libpng} követelménye a \textit{zlib}. 
    \item SDL\_net (opcionális)
    
    Lehetővé teszi a hálózat használatát. A többjátékos mód eléréséhez elengedhetetlen, ugyanakkor nincs tervben ennek a használata. Amennyiben ez módosul, ezt is hozzáadjuk a könyvtárakhoz.
    \item SDL\_sound (opcionális)
    
    Lejátszhatóak lesznek a tömörített hanggal rendelkező CD-k.
    \item PDCurses (opcionális)
    
    Elérhetővé teszi a DOS Debuggert.

\end{itemize}

Miután eldöntöttük mely könyvtárak szükségesek nekünk a fordításhoz, a fordítási útmutató alapján beállítjuk a linkereket megfelelően, és a szükséges DLL fájlokat is. 

Utolsó dolgunk átírni a \textit{config.h} nevű fájlban az alábbi konfigurációkat: 

\begin{cpp}

    ...
    #define C_DEBUG 0 // PDCurses

    #define C_SSHOT 1 // libpng

    #define C_SRECORD 1 // zlib

    #define C_MODEM 0 // SDL\_net 

    #define C_IPX 0 // SDL\_net 
    ...
\end{cpp}
