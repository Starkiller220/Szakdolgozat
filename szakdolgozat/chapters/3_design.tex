\Chapter{Felhasznált eszközök és technológiák}

\Section{Dosbox}

A szoftver futtatásához elengedhetetlen egy DOS emulátor, amire a legjobb választás a nyílt forráskódú Dosbox. Az SDL multimédiás könyvtárat használja, és a lehető leghűbben lehet vele Dos-os környezetet emulálni. A nyílt forráskódnak köszönhetően betekintést lehet nyerni a működésébe, továbbá módosítani saját igényünknek megfelelően. 

\Section{OpenCv}

Az OpenCv (teljes nevén Open Source Computer Vision Library) a nevéből adódóan egy nyílt forráskódú gépi látás és tanulás könyvtár, mely elérhető számos nyelven. Azért jött létre, hogy közös infrastruktúrát biztosítson gépi látáshoz és tanuláshoz, ezáltal felgyorsítsa az ezt alkalmazó programok fejlődését, és megsokszorozza a számát. Manapság a könyvtár több mint 2500 algoritmust tartalmaz és folyamatosan bővül.

\Section{ZeroMQ}

A ZeroMQ egy beágyazható hálózati könyvtár, mely úgy működik mint egy párhuzamossági keretrendszer. Olyan socketeket biztosít, amelyek atomi üzeneteket szállítanak különböző transzportokon keresztül, például folyamaton belül, folyamatok között, TCP-n és multicastban. A socketeket N-ről-N-re összekapcsolhatjuk olyan mintákkal, mint a fan-out, pub-sub, task distribution és request-reply. Elég gyors ahhoz, hogy a klaszteres termékek szöveteként szolgáljon. Aszinkron I/O modellje aszinkron üzenetfeldolgozási feladatokként felépített, skálázható többmagos alkalmazásokat biztosít. Számos nyelvi API-val rendelkezik, és a legtöbb operációs rendszeren fut. Ez hatalmas előny 2 különböző nyelven írt program összekapcsolásánál, és pont ez az ami miatt ezt a könyvtárat választottam.

\Section{GIMP}

A GIMP (teljes nevén GNU Image Manipulation Program) egy nyílt forráskódú és ingyenes képszerkesztő mely több platformra is elérhető. Tudása megközelíti, és néhol le is körözi a fizetős társát, a Photoshop-ot. A szakdolgozat elkészítéséhez azért volt erre a szoftverre szükségem, mert nem minden warcraft-os sprite volt fellelhető az interneten, így sokakat nekem kellett kinyerni a játékról készített képernyőképekből. Ezt a folyamatot később részletezni is fogom. Számos harmadik féltől származó bővítés található hozzá, melyek közül kettő a hasznomra is vált (Save all és BIMP).
% ref: https://antumdeluge.wordpress.com/2017/12/20/export-all-open-images-in-gimp/
% ref: https://alessandrofrancesconi.it/projects/bimp/

\iffalse
\Section{Python}

asd

\Section{Numpy}

asd

\Section{Pyteserract}

asd
\fi
