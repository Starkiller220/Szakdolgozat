\Chapter{Összefoglalás}

A szakdolgozatom célja egy Warcaftot játszó mesterséges intelligencia létrehozása volt, amely úgy gondolom sikerült is. Természetesen vannak olyan futtatások ahol az MI elég csúnya vereséget szenved, ugyanakkor van olyan eset is ahol egész sokáig bírja. Sajnos győzelmet a sok próbálkozás során sem sikerült kicsikarni belőle, ugyanakkor személy szerint úgy gondolom hogy élvezetes nézni ahogyan próbálkozik játszani. 

Végezetül elmondható, hogy a program igenis próbálkozik legyőzni ellenfelét, több kevesebb sikerrel. Legszembetűnőbb problémát az építés okozza, amely sajnos igen ritkán aktiválódik a DosBox okozta limitációk miatt, legyen akármennyire is jó az algoritmus. Ugyanakkor ami az egyik esetben hátrány, az valahol igenis előny, mivel egy kevésbé input igényes alkalmazás esetén igazán hasznos dolgokat is el lehet érni egy ilyen kliens-szerver szerű alkalmazással.
Ahhoz viszont hogy ezt az intelligenciát tovább lehessen fejleszteni, a DosBox input rendszerét eléggé át kellene tervezni, ami lévén hogy kevesen használják a programot, nem feltétlenül járható út. Ugyanakkor érdemes lehet a már említett más emulátorokkal is kísérletezni, akár automatizálni akár szintén egy mesterséges intelligenciát írni egy játékra. Mivel az utóbbi időkben megnőtt az igény a különböző emulátorokra, főként az aktuális és az azt megelőző konzolgenerációs játékok személyi számítógépen történő futtatására, így úgy érzem aktuális lehet ezekkel komolyabban foglalkozni, azonban mivel ezek a szoftverek jelenleg nagyon erőforrás igényesek, egy darabig még nem igazán fogunk tapasztalni ilyesféle projekteket.